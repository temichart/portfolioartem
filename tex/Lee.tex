\documentclass{article}
\usepackage[utf8x]{inputenc} % Включаем поддержку UTF8
\usepackage[russian]{babel} % Включаем пакет для поддержки русского языка

\setcounter{page}{1}

\title{Курсовая работа}
\author{Ли Артём Владиславович}
\date{18.03.2021 г.}

\usepackage{natbib}
\usepackage{graphicx}
\usepackage{indentfirst}
\linespread{1.5}
\usepackage[verbose]{geometry}
\geometry{top=25pt, left=70pt, right=70pt,textwidth=450pt,textheight=900pt,heightrounded}

\begin{document}
	\maketitle
	
	\section{Разработка интернет магазина для продажи хозяйственных товаров}
	
	\subsection{Технико-экономическая характеристика объекта исследования}
	Полное название: Магазин хозяйственных товаров «Пенка». Сокращенное название: Магазин «Пенка».
	
	Магазин «Пенка» зарегистрирован в соответствии с законом Российской Федерации «О государственной регистрации юридических лиц и индивидуальных предпринимателей».
	Деятельность организации связана с продажей хозяйственных и бытовых товаров.
	
	Магазин «Пенка» имеет линейную структуру управления, которая показана на рис. 1. Одно лицо управляет всем набором указаний, а подчиненные выполняют распоряжение только одного руководителя. Директору магазина подчиняются: бухгалтер, товаровед и продавец, а им в свою очередь подчиняются исполнители. 
	\begin{figure}[h]
		\center{\includegraphics{img1.png}}
		\caption{Организационная структура магазина хозяйственных товаров.}
	\end{figure}
	\newpage
	
	\subsection{Выбор проектного решения}
	В современных реалиях перед web – разработчиками стоит множество задач: начиная от разработки простых развлекательных сайтов, заканчивая созданием полноценных, серьёзных проектов, которые должны быть хорошо защищены от несанкционированного доступа третьих лиц. 
	
	Для реализации таких проектов требуется хорошо ознакомиться с технологиями, которые предлагает интернет рынок на сегодняшний день. 
	
	Следовательно, чтобы создать полноценный функционирующий интернет магазин web – разработчику потребуется знать не только язык программирования, на котором он будет разрабатывать проект, но и выбрать подходящий фреймворк или систему управления контентом (cms).
	$$
		x_{1,2}=\frac{-b\pm \sqrt{b^2-4ac}}{2a}
	$$
	
	\newpage
	\tableofcontents
\end{document}
	